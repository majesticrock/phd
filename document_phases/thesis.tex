%%%%%%%%%%%%%%%%%%%%%%%%%%%%%%%%%%%%%%%%%%%%%%%%%%%%%%%%%%%%%%%%%%%%%%%%%%%%%%%%
%%%%%%%%%%%%%%%%%%   Vorlage für eine Abschlussarbeit   %%%%%%%%%%%%%%%%%%%%%%%%
%%%%%%%%%%%%%%%%%%%%%%%%%%%%%%%%%%%%%%%%%%%%%%%%%%%%%%%%%%%%%%%%%%%%%%%%%%%%%%%%

% Erstellt von Maximilian Nöthe, <maximilian.noethe@tu-dortmund.de>
% ausgelegt für lualatex und Biblatex mit biber

% Kompilieren mit
% latexmk --lualatex --output-directory=build thesis.tex
% oder einfach mit:
% make

\documentclass[
  %tucolor,       % remove for less green,
  BCOR=12mm,     % 12mm binding corrections, adjust to fit your binding
  parskip=half,  % new paragraphs start with half line vertical space
  open=any,      % chapters start on both odd and even pages
  cleardoublepage=plain,  % no header/footer on blank pages
]{tudothesis}


% Warning, if another latex run is needed
\usepackage[aux]{rerunfilecheck}

% just list chapters and sections in the toc, not subsections or smaller
\setcounter{tocdepth}{1}

%------------------------------------------------------------------------------
%------------------------------ Fonts, Unicode, Language ----------------------
%------------------------------------------------------------------------------
\usepackage{fontspec}
\defaultfontfeatures{Ligatures=TeX}  % -- becomes en-dash etc.

% load english
% the main language has to come last
\usepackage[english]{babel}

% intelligent quotation marks, language and nesting sensitive
\usepackage[autostyle]{csquotes}

% microtypographical features, makes the text look nicer on the small scale
\usepackage{microtype}

%------------------------------------------------------------------------------
%------------------------ Math Packages and settings --------------------------
%------------------------------------------------------------------------------

\usepackage{amsmath}
\usepackage{amssymb}
\usepackage{mathtools}

% Enable Unicode-Math and follow the ISO-Standards for typesetting math
\usepackage[
  math-style=ISO,
  bold-style=ISO,
  sans-style=italic,
  nabla=upright,
  partial=upright,
  warnings-off={mathtools-colon,mathtools-overbracket}, % suppress some unnecessary warnings
]{unicode-math}
\setmathfont{Latin Modern Math}

% nice, small fracs for the text with \sfrac{}{}
\usepackage{xfrac}


%------------------------------------------------------------------------------
%---------------------------- Numbers and Units -------------------------------
%------------------------------------------------------------------------------

\usepackage[
  locale=DE,
  separate-uncertainty=true,
  per-mode=symbol-or-fraction,
]{siunitx}

%------------------------------------------------------------------------------
%-------------------------------- tables  -------------------------------------
%------------------------------------------------------------------------------

\usepackage{booktabs}       % \toprule, \midrule, \bottomrule, etc

%------------------------------------------------------------------------------
%-------------------------------- graphics -------------------------------------
%------------------------------------------------------------------------------

\usepackage{graphicx}
% currently broken
% \usepackage{grffile}

% allow figures to be placed in the running text by default:
\usepackage{scrhack}
\usepackage{float}
\floatplacement{figure}{htbp}
\floatplacement{table}{htbp}

% keep figures and tables in the section
\usepackage[section, below]{placeins}

% allows to include PDFs as full pages
\usepackage{pdfpages}

% Set the PDF Version of this document to 1.7 (1.4 is the current default)
% This is needed so that PDFs with Version >1.5 can be included
\pdfvariable minorversion=7

%------------------------------------------------------------------------------
%---------------------- customize list environments ---------------------------
%------------------------------------------------------------------------------

\usepackage{enumitem}

%------------------------------------------------------------------------------
%------------------------------ Bibliographie ---------------------------------
%------------------------------------------------------------------------------

\usepackage[
style=phys, autolang=hyphen,
articletitle=true, biblabel=brackets,
chaptertitle=false, pageranges=false
]{biblatex}
\addbibresource{references.bib}  % the bib file to use
\DefineBibliographyStrings{german}{andothers = {{et\,al\adddot}}}  % replace u.a. with et al.


% Last packages, do not change order or insert new packages after these ones
\usepackage[pdfusetitle, unicode, linkbordercolor=tugreen, citebordercolor=tugreen]{hyperref}
\usepackage{bookmark}
\usepackage[shortcuts]{extdash}
\usepackage{chemformula}
\usepackage{braket}
\usepackage{subdepth}

\newcommand{\dt}{
  \ensuremath{\frac{\text{d}}{\text{d}t}}
}

%------------------------------------------------------------------------------
%-------------------------    Angaben zur Arbeit   ----------------------------
%------------------------------------------------------------------------------

\author{Joshua Althüser}
\title{Charge density wave and superconducting phases in the attractive Hubbard model}
\date{2022}
\birthplace{Soest}
\chair{Lehrstuhl für Theoretische Physik Ia}
\division{Fakultät Physik}
\thesisclass{Phd}
\submissiondate{N/A}
\firstcorrector{Prof.~Dr.~Götz S. Uhrig}

% tu logo on top of the titlepage
\titlehead{\includegraphics[height=1.5cm]{logos/tu-logo.pdf}}

\begin{document}
\frontmatter
\maketitle

% Gutachterseite
\makecorrectorpage

% hier beginnt der Vorspann, nummeriert in römischen Zahlen
%\input{content/00_abstract.tex}
\tableofcontents

\mainmatter
% Hier beginnt der Inhalt mit Seite 1 in arabischen Ziffern
\input{content/01_introduction.tex}
\chapter{Model}
\label{chap:model}

This work focuses on the extended Hubbard model on a 2D square lattice with periodic boundary conditions.
Its general Hamiltonian in position space is given by

\begin{equation}
    \begin{aligned}
        H_\text{H} =&-t \sum_{\langle r,s \rangle,\sigma} \left( c_{r\sigma}^\dagger c_{s\sigma} + \text{h.c.} \right) 
                        + U \sum_r n_{r\uparrow} n_{r\downarrow} \\
                    &+ \frac{V}{2} \sum_{\langle r,s \rangle} \left( n_{r\uparrow} + n_{r\downarrow} \right) 
                        \left( n_{s\uparrow} + n_{s\downarrow} \right)\,,
    \end{aligned}  
\end{equation}

where $c_{r\sigma}^{(\dagger)}$ represents the standard fermionic annihilation (creation) operator 
and $n_{r\sigma}$ the number operator with spin $\sigma$ on the lattice site $r$.
The constants $t$, $U$, and $V$ govern the strength of the nearest neighbour hopping, 
the onsite interaction, and the nearest neighbour interaction, respectively.
Applying a Fourier transform yields the new momentum space Hamiltonian

\begin{equation}
    \begin{aligned}
        H_k =& \sum_{k \sigma} \epsilon_0 (k) n_{k\sigma} 
                + \frac{U}{N} \sum_{k,p,q} c_{k\uparrow}^\dagger c_{k+q\uparrow} c_{p\downarrow}^\dagger c_{p-q\downarrow} \\
            &+ \frac{V}{2N} \sum_{k,p,q} \sum_{\sigma, \sigma'} \sum_{\delta} \cos(q\delta) 
                    c_{k\sigma}^\dagger c_{k+q\sigma} c_{p\sigma'}^\dagger c_{p-q\sigma'}\,.
    \end{aligned} 
\end{equation}

Here, $k$, $p$, and $q$ are momenta, $\delta$ references the vectors connecting nearest neighbours within the lattice,
and $N$ is the total number of lattice sites.
The diagonal part is given by

\begin{equation}
    \epsilon_0 (k) := -2t \sum_\delta \cos(\delta k)\,.
\end{equation}
\input{content/03_methods.tex}
\input{content/04_results.tex}
\input{content/05_conclusion.tex}

\appendix
% Hier beginnt der Anhang, nummeriert in lateinischen Buchstaben
%\input{content/a_anhang.tex}

\backmatter
\printbibliography

\cleardoublepage
% From https://www.tu-dortmund.de/studierende/im-studium/pruefungsangelegenheiten/allgemeine-vordrucke/
%\includepdf[pages={1}]{content/eides_statt.pdf}

\end{document}
