\documentclass[20pt]{article}

% set margins and ignore intends
\usepackage[margin=0.9in]{geometry}
\setlength{\parindent}{0pt}

% include graphics
\usepackage{graphicx}

% math stuff
\usepackage{amsmath}
\usepackage{physics}
\usepackage{bbm}

% colors and refs
\usepackage{color}
\usepackage{xcolor}
\usepackage{hyperref}
\usepackage{url}

% definitions and commands
\xdefinecolor{tugreen}{RGB}{132, 184, 24} % define tu green
\def\blue#1{{\color{blue}{#1}}}
\newcommand{\mchapter}[1]{ \begin{center} \color{tugreen!95!black} \LARGE \bf \textbf{#1} \end{center}} % self-made chapters
\newcommand{\msection}[1]{ { \vspace{5mm} \hspace{-6mm} \large \textbf{#1}} \vspace{2mm} } % self-made sections
\newcommand{\mb}{\mathbf} % bold math symbols
\newcommand{\mcaption}[2]{ \\ Figure {#1}: {#2} } % self-made figure captions

% head
% definitions
\font\fb=cmss12   scaled 1200
\font\fc=cmss10   scaled 1000

% head of article
\def\myhead
{

% TU Logo
\vspace*{-1.9cm}
{
    \includegraphics[width=72mm]{tudo_logo.pdf}
}

% CMT 
\vspace*{0.2cm}
{
    \fb
    Condensed Matter Theory\\[1mm]
    Prof.\ Dr.\ G\"otz S. Uhrig\\
}

% address
\vspace*{-3.7cm}
{
    \flushright
    \fc
    \vspace*{7mm}
    TU Dortmund University\\
    \vspace*{-0.6mm}
    Department of Physics / Uhrig Group \\
    \vspace*{-0.6mm}
    44221 Dortmund\\
    \vspace*{-0.6mm}
    Germany\\
    \vspace*{-0.6mm}
    Tel: +49 231 755 3547\\
    \vspace*{-0.6mm}
    Mail: goetz.uhrig@tu-dortmund.de\\
}

}

%%%%%%%%%%%%%%%%%%%%%%%

%%%%%% version 2 (müsste noch umgeschrieben werden) %%%%%%:

%%\vspace*{-0.7cm}
%%\hspace*{-1cm}
%\includegraphics[width=72mm]{tudo_logo.eps}
%%\vspace*{-2.5cm}
%
%
%\fb
%Condensed Matter Theory\\[1mm]
%Prof.\ Dr.\ G\"otz S. Uhrig\\
%\fc \flushright
%Tel/Fax: +49 (0)231 755-3547\,/\,5059\\
%\vspace*{-1mm}
%goetz.uhrig@tu-dortmund.de\\
%}
%
%%\signature{Prof.\ Dr.\ G\"otz S. Uhrig}
%%\place{Dortmund}
%\address{\myhead}
%\date{} % will be ignored
%%\phone{+49 (0)231}{755-3547}
%\backaddress{Condensed Matter Theory, TU Dortmund, D-44221 Dortmund}

%%%%%%%%%%%%%%%%%%%%%%%

\endinput
%%


\newcommand{\expec}[1]{\langle #1 \rangle}
\newcommand{\timeDeriv}{\frac{\mathrm{d}}{\mathrm{d}t}}

\begin{document}

% Wer eine Schrift mit Serifen verwendet, kriegt vom Designer
% persönlich auf die Rübe. Darum \sf. 
\sf

% head:
\myhead

% title:
\vspace{2mm}
\mchapter{Collective excitations in correlated quantum materials}
\vspace{-3mm}

% text:
\msection{Scientific context}

Correlated quantum materials have been an ever-present subject of scientific research since the discovery of quantum mechanics.
The BCS-theory of superconductivity\cite{theory_of_sc} from 1957 was the first of its kind to be able to describe the phenomenon.
It explains that two electrons form so-called Cooper pairs, which - in contrary to the composing electrons - are bosonic quasiparticles.
This pair formation is driven by electron-phonon interactions, that yield an effective attractive electron-electron interaction.

If the temperature is below some critical temperature $T_c$ they can condense into a macroscopic occupation of the ground state and allow for the famous perfect conductivity.
The two prominent collective excitations in a superconductor are the so-called Higgs and phase modes \cite{higgs, bo22}.
The former is energetically located at the band edge and sometimes referred to as the amplitude mode.
The latter originates in the invariance of the complex phase. 
Changing the phase does not affect the energy, thus the mode is located at 0.

A well-known and often used model is the so-called Hubbard model. Its Hamiltonian reads
\begin{equation}
    \label{eqn:hubbard}
    H = -t \sum_{\langle i, j \rangle, \sigma} c_{i\sigma}^\dagger c_{j\sigma}  + U \sum_i c_{i\uparrow}^\dagger c_{i\downarrow}^\dagger c_{i\downarrow} c_{i\uparrow}\,,
\end{equation}
where the $\langle i, j \rangle$ denotes the summation over next neighbours, $t$ is a hopping strength, 
$U$ denotes the on-site interaction strength and $c_{i\sigma}^{(\dagger)}$ annihilates (creates) an electron with spin $\sigma$ on lattice site $i$.
Most commonly, $U$ is positive and represents the repulsive Coulomb interaction of two electrons.
In this case, the model exhibits antiferromagnetism (AFM).

However, negative values for $U$ are also possible \cite{sentef}. They are oftentimes achieved by the aforementioned effective interactions mediated by phonons.
If the system is precisely at half-filling and the underlying lattice of the system is bipartite, e.g. a square or simple cubic lattice, 
the system can be transformed by a specific kind of particle-hole transformation \cite{micnas90}.
This transformation effectively flips the sign of $U$, allowing researches to use the previously known antiferromagnetic results.
However, instead of antiferromagnetism, the system now exhibits the coexistence of two distinct phases.
Namely, s-wave superconductivity as described by the aforementioned BCS-theory and a so-called charge-density wave (CDW).
The latter describes a system, where the electron density exhibits a wave like behaviour, similar to the spins in an antiferromagnetic phase.


\msection{Methodology: Mean field theory and iterated equations of motion}

Handling a quartic Hamiltonian, such as \autoref{eqn:hubbard}, is practically very difficult up to impossible,
as the Hilbert spaces grow exponentially with the system size.
The mean field approximation aims to reduce those kinds of Hamiltonians to a bilinear form by neglecting fluctuations around the the equilibrium.
Therefore, one introduces so-called normal ordered operators
\begin{equation}
    :A: = A - \expec{A}_0\,.
\end{equation}
These allow rewriting the quartic term in the Hubbard Hamiltonian to
\begin{subequations}
\begin{align}
    c_{i\uparrow}^\dagger c_{i\downarrow}^\dagger c_{i\downarrow} c_{i\uparrow} &= :c_{i\uparrow}^\dagger c_{i\downarrow}^\dagger c_{i\downarrow} c_{i\uparrow}:\\
        &+\expec{c_{i\uparrow}^\dagger c_{i\downarrow}^\dagger} :c_{i\downarrow} c_{i\uparrow}: 
            + :c_{i\uparrow}^\dagger c_{i\downarrow}^\dagger: \expec{c_{i\downarrow} c_{i\uparrow}} \\
        &-\expec{c_{i\uparrow}^\dagger c_{i\downarrow}} :c_{i\downarrow}^\dagger c_{i\uparrow}: 
            - :c_{i\uparrow}^\dagger c_{i\downarrow}: \expec{c_{i\downarrow}^\dagger c_{i\uparrow}} \\
        &+\expec{c_{i\uparrow}^\dagger c_{i\uparrow}} :c_{i\downarrow}^\dagger c_{i\downarrow}: 
            + :c_{i\uparrow}^\dagger c_{i\uparrow}: \expec{c_{i\downarrow}^\dagger c_{i\downarrow}} + \text{const}\,,
\end{align}
\end{subequations}
where the first term $:c_{i\uparrow}^\dagger c_{i\downarrow}^\dagger c_{i\downarrow} c_{i\uparrow}:$ describes only small fluctuations that are neglected.
The remainder is only bilinear in terms of operators as the expectation values are merely numbers.

The challenge here is obtaining these expectation values as they depend on the mean-field Hamiltonian, which in turn depends on the expectation values.
The solution to this is a self-consistency calculation, for which one selects some starting values,
solves the Hamiltonian for the new starting values and iterates until convergence is achieved.

Afterwards, we use the iterated-equations-of-motion method (iEoM) to analyse the system for occurring excitations \cite{philip18, Kalthoff2017}.
This method is based on the Heisenberg equations of motion 
\begin{equation}
    \timeDeriv A(t) = \frac{i}{\hbar} [H, A(t)]\,.
\end{equation}
To solve this equation, one needs an operator basis, which is complete in the sense, that each commutator $[H, A(t)]$ can be represented by this basis.
Practically, this is often requires an infinite basis, which is impossible to implement numerically.
The idea of the iEoM is, to iterate the commutation process and truncate at some point, assuming that the error becomes smaller with each iteration.

Furthermore, an operator scalar product $(A|B)$ is introduced, which allows reducing the equations of motion from the operator form above to
\begin{equation}
    \timeDeriv \mathbf{c}(t) = \mathcal{M} \mathbf{c}(t)\,,
\end{equation}
where $\mathbf{c}(t)$ is a vector of time-dependent complex \emph{numbers} and a matrix
\begin{equation}
    \mathcal{M}_{ij} := (A_i | [H, A_j])\,,
\end{equation}
where $A_i, A_j$ are the operators from the aforementioned operator basis.

All system dynamics are now contained within the complex-valued matrix $\mathcal{M}$, which can be analysed using a plethora of well-established numerical methods.


\msection{Possible tasks during an internship}

Generally, the internship involves mainly \emph{numerical} tasks and \emph{coding}, mainly utilising C++ and Python.
Your project will involve the investigation of the excitations within a Hubbard model, depending on interactions or different lattices.
You will begin by familiarising yourself with the topic using literature that we will provide.
Subsequently, you would write a program that solves a self-consistency problem and analyses the system dynamics.
We will provide you with some of our existing code fragments to help you get started.
Generally, some experience in \textbf{C++}, \emph{Python} or coding is advantageous. Moreover, knowledge of quantum mechanics is required.


\msection{General information}

The chair currently consists of 9 people, that is, 6 PhD students, 1 postdoc as well as Prof. G\"otz S. Uhrig and Prof. Joachim Stolze. 
The group works on a broad range of physical topics such as topological magnonics, non-equilibrium physics, coherence control and more. 
We have a weekly seminar in which progress is reported and articles are discussed (Journal Club). Feel free to visit our homepage at \url{https://cmt.physik.tu-dortmund.de/uhrig-group/}. 
Our group already successfully hosted 2 RISE students in 2023.
The methodology and the investigation of collective excitations is the research area of the PhD student Joshua Alth\"user who will also be your supervisor.
You will get your own office space and access to our compute clusters. We do not offer a virtual internship.

% bibliography
\bibliography{lit} 
\bibliographystyle{ieeetr}


\end{document}

