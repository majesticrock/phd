\documentclass[
    reprint, 
    aps,
    preprintnumbers,
    twocolumn,
    prb,
    superscriptaddress
]{revtex4-2}


%=======================
% Packages:
%=======================

\usepackage[utf8]{inputenc}
\usepackage{booktabs}
\usepackage[multiple]{footmisc}
\usepackage{lipsum}
\usepackage{rotating}
\usepackage{perpage}
\usepackage{chronology}
\usepackage{lipsum}
\usepackage{amssymb}
\usepackage{amsbsy}
\usepackage{amsmath}
\usepackage{tikz}
\usepackage[T1]{fontenc}
\usepackage{etoolbox}
\usepackage{graphics}
\usepackage{abraces}
\makeatletter
\let\unit\relax % otherwise there are package conflicts with siunitx
\makeatother
\usepackage{siunitx}
\usepackage{hyperref}
%\usepackage{float}
\usepackage{multirow}
\usepackage{mathtools}
\usepackage{bm}
\usepackage{url}
\usepackage{physics}
\usepackage{hyperref}
\usepackage{bbm}
\usepackage{color}
\usepackage{placeins}
\usepackage[normalem]{ulem}
\usepackage{mhchem}


%=======================
% New Commands:
%=======================

\newcommand{\vk}{\vec{k}}
\newcommand{\vQ}{\vec{Q}}
\newcommand{\up}{\uparrow}
\newcommand{\down}{\downarrow}
\newcommand{\kplusQ}{\vk+\vQ}
\newcommand{\kminusQ}{-\vk-\vQ}

\begin{document} 

% Preliminary title
\title{Collective excitations in the 2- and 3-dimensional extended Hubbard model}

%=======================
% Authors:
%=======================

\author{Joshua Alth\"user}\email{joshua.althueser@tu-dortmund.de}
\affiliation{Condensed Matter Theory, TU Dortmund University,
Otto-Hahn Stra\ss{}e 4, 44221 Dortmund, Germany}

\author{G\"otz S.~Uhrig}
\email{goetz.uhrig@tu-dortmund.de}
\affiliation{Condensed Matter Theory, TU Dortmund University,
Otto-Hahn Stra\ss{}e 4, 44221 Dortmund, Germany}

\date{\today}

%%%%%%%%%%%%%%%%%%%%%%%%%%%%%%%%%%%%%%%%%%%%%%%%%%%%%%%%%%%%%%%%%%%%%%%%%%%%%%%%%%%%%%%%%%%%%%%%%%%%%%%%%%%%%%%%%%%%%
%%%%%%%%%%%%%%%%%%%%%%%%%%%%%%%%%%%%%%%%%%%%%%%%%%%%%%%%%%%%%%%%%%%%%%%%%%%%%%%%%%%%%%%%%%%%%%%%%%%%%%%%%%%%%%%%%%%%%
%%%%%                                                  Abstract                                                 %%%%%
%%%%%%%%%%%%%%%%%%%%%%%%%%%%%%%%%%%%%%%%%%%%%%%%%%%%%%%%%%%%%%%%%%%%%%%%%%%%%%%%%%%%%%%%%%%%%%%%%%%%%%%%%%%%%%%%%%%%%
%%%%%%%%%%%%%%%%%%%%%%%%%%%%%%%%%%%%%%%%%%%%%%%%%%%%%%%%%%%%%%%%%%%%%%%%%%%%%%%%%%%%%%%%%%%%%%%%%%%%%%%%%%%%%%%%%%%%%
\begin{abstract}
Abstract text Abstract text Abstract text Abstract text Abstract text Abstract text Abstract text 
\end{abstract}

\maketitle

%%%%%%%%%%%%%%%%%%%%%%%%%%%%%%%%%%%%%%%%%%%%%%%%%%%%%%%%%%%%%%%%%%%%%%%%%%%%%%%%%%%%%%%%%%%%%%%%%%%%%%%%%%%%%%%%%%%%%
%%%%%%%%%%%%%%%%%%%%%%%%%%%%%%%%%%%%%%%%%%%%%%%%%%%%%%%%%%%%%%%%%%%%%%%%%%%%%%%%%%%%%%%%%%%%%%%%%%%%%%%%%%%%%%%%%%%%%
%%%%%                                                Introduction                                               %%%%%
%%%%%%%%%%%%%%%%%%%%%%%%%%%%%%%%%%%%%%%%%%%%%%%%%%%%%%%%%%%%%%%%%%%%%%%%%%%%%%%%%%%%%%%%%%%%%%%%%%%%%%%%%%%%%%%%%%%%%
%%%%%%%%%%%%%%%%%%%%%%%%%%%%%%%%%%%%%%%%%%%%%%%%%%%%%%%%%%%%%%%%%%%%%%%%%%%%%%%%%%%%%%%%%%%%%%%%%%%%%%%%%%%%%%%%%%%%%

\section{Introduction}\label{sec:introduction}

The Hubbard model has been employed in a plethora of previous studies. 
Early studies proved the existence of eigenstates to the Hubbard Hamiltonian that exhibit off-diagonal long-range order,
tantalizing the model's usage for the description of high-temperature superconductivity \cite{yang89}.
Shortly afterward, an exact $SO(4)$ symmetry was discovered,
which allows a degeneracy of superconductivity (SC) and a charge-density wave (CDW) governed by an attractive on-site interaction \cite{yang90}.
This coexistence can be argued by the existence of a particle-hole transformation on bipartite lattices, 
that maps the attractive Hubbard model exactly onto the repulsive one, exhibiting antiferromagnetism (AFM) \cite{Hirsch85}.
The previously mentioned SC and CDW phases map to different spin operators \cite{zitko15,lieb89}.
\newline
Many recent experimental and theoretical studies examined the driven systems in the hope of inducing superconductivity 
\cite{Nicoletti14,Krull14,Moor14,Casandruc15,patel16,sentef17,Buenemann17}.
Other studies, focusing on equilibrium systems, 
investigated the phases and various quantities therein of said model including various additional interactions with and without doping 
\cite{Micnas88,Micnas88b,Micnas89,Dzierzawa92,Kostyrko92,Eriksson95,Staudt00,Onari04,Toschi05,Brackett16,Paki19,romer20,Sushchyev22}.
In this paper, we will restrict ourselves to the half-filled Hubbard model, including an additional intersite interaction, on a square and a simple cubic lattice.
\textcolor{red}{Für den 3D Fall finde ich keine Phasendiagramme des extended Hubbard models, ggf nochmal suchen...}
The 2D case already exhibits a wide range of possible phases, including CDW, AFM as well as $s$- and $d_{x^2 - y^2}$-wave superconductivity
\cite{Micnas88b,Tsuchiura95,Su01,Su04,ha11,Huang13,Jiang22}.
\newline
In this article, we will first employ a mean-field approximation to the interaction terms 
and then use the methodology from the so-called iterated equations of motion approach,
which has already seen success in the handling of interaction quenches \cite{uhrig09,hamerla13,hamerla14,bleicker18}.
Here, one commutes operators from a suitable operator basis with the Hamiltonian and adds the newly occurring terms to it.
Naturally, a truncation is necessary for most practical applications, however, the approximation becomes better the more terms are included within the basis.
The functionality of this method has been compared to the density matrix formalism and has proved to be considerably more accurate \cite{Kalthoff17}.
Additionally, we will show an explicit way to obtain various Green's functions from the aforementioned methodology.
By extension, we also obtain the spectral functions of the investigated systems and discuss the excitations found therein.
The prominent examples in the SC phase are the well-known phase (Leggett) and amplitude (Higgs) modes \cite{Cea14,Krull16,Schwarz20,Fan22}.
\newline
The remainder of this paper is organized as follows:
In \autoref{sec:model} we will introduce the model and its Hamiltonian as well as the main equations and relations used throughout this article.
Next, we show our results in \autoref{sec:results}.
Lastly, in \autoref{sec:conclusion}, we will discuss our results.
In the appendix \textcolor{red}{(or in the supplemental material?)}, we will derive and explain our methods in detail.

\section{Model and governing equations}\label{sec:model}

The Hamiltonian of the extended Hubbard model is given by
\begin{equation}
    \begin{aligned}
        H = &-t \sum_{\langle i, j \rangle, \sigma} \left( c_{i\sigma}^\dagger c_{j\sigma} + \text{h.c.} \right) 
        + \mu \sum_{i,\sigma} n_{i\sigma} \\
        &+ U \sum_{i} n_{i\uparrow} n_{i\downarrow} 
        + \frac{V}{2} \sum_{\langle i, j\rangle, \sigma} n_{i\sigma} n_{j\sigma}\,,
    \end{aligned}
\end{equation}

where $c_{i\sigma}^{(\dagger)}$ annihilates (creates) an electron with spin $\sigma$ on lattice site $i$ and $\langle i, j\rangle$ denotes the summation over next neighbours.
The parameters are the hopping amplitude $t$, the onsite interaction $U$, the intersite interaction $V$ and the chemical potential $\mu$.
Throughout this article, we will give all energy-related quantities in terms of the hopping amplitude, which we subsequently set to unity $t=1$.
Applying a Fourier transformation into $k$-space yields the single-particle dispersion 

\begin{equation}
    \epsilon_0 (\vk) = -2t \sum_{\alpha=1}^D \cos(k_\alpha)\,,\quad k_\alpha \in [-\pi, \pi)\,,
\end{equation}
with the system's dimension $D$ and the dimensionless momentum $\vk$.
The interaction terms are then mean-field decoupled. 
We follow the naming scheme of Ref. \cite{sentef17}

\begin{equation}
    \begin{aligned}
        n_{k\sigma} &:= c_{\vk\sigma}^\dagger c_{\vk\sigma}\,,      &f_k     &:= c_{-\vk\down} c_{\vk\up}\,, \\
        g_{k\sigma} &:= c_{\vk\sigma}^\dagger c_{\vk+\vQ\sigma}\,,  &\eta_k  &:= c_{-\vk-\vQ\down} c_{\vk\up}\,,
    \end{aligned}
\end{equation}

where $\vQ := (\pi, ...)$ defines the nesting vector for the CDW and AFM phases.
We use these abbreviations to write down the mean-field parameters

\begin{subequations}
    \begin{align}
        \Delta_\text{CDW} &= \left(\frac{U}{2N} - \frac{ZV}{N}\right) \sum_{\vk\sigma} \langle g_{\vk\sigma} \rangle\,, \\
        \Delta_\text{AFM} &= \frac{U}{2N} \sum_{\vk} \left( \langle g_{k\uparrow} \rangle - \langle g_{\vk\downarrow} \rangle \right)\,, \\
        \Delta_\text{SC} &= \frac{U}{N} \sum_{\vk} \langle f_{\vk} \rangle\,, \\
        \Delta_\eta &= \frac{U}{N} \sum_{\vk} \langle \eta_{\vk} \rangle\,, \\
        \Delta_n &= \frac{V}{N} \sum_{\vk,\sigma} \sum_{\alpha=1}^D \cos k_\alpha \langle n_{\vk\sigma} \rangle\,,
    \end{align}
\end{subequations}

where $Z$ denotes the coordination number of the lattice.
In total, we obtain the mean-field Hamiltonian in spinor representation as

\begin{equation}
    H_\text{MF} = \sum_{\vk} \Psi^\dagger (\vk) h(\vk) \Psi (\vk)\,,
\end{equation}

with the spinors
\begin{equation}
    \Psi^\dagger (\vk) := \left( c_{\vk\up}^\dagger\,, c_{\kplusQ\up}^\dagger\,, c_{-\vk\down}\,, c_{\kminusQ\down} \right)
\end{equation}

and 

\begin{widetext}
\begin{equation}
    h(\vk) := \begin{pmatrix}
        \epsilon_0 (\vk) & \Delta_\text{CDW}^* - \Delta_\text{AFM}^* & \Delta_\text{SC} & \Delta_\eta \\
        \Delta_\text{CDW} - \Delta_\text{AFM} & \epsilon_0 (\vk + \vQ) & \Delta_\eta & \Delta_\text{SC} \\
        \Delta_\text{SC}^* & \Delta_\eta^* & - \epsilon_0 (-\vk) & - \Delta_\text{CDW} - \Delta_\text{AFM} \\
        \Delta_\eta^* & \Delta_\text{SC}^* & - \Delta_\text{CDW}^* - \Delta_\text{AFM}^* & - \epsilon_0 (-\vk - \vQ)\,.
        \end{pmatrix}
\end{equation}
\end{widetext}

Now, the entire Hamiltonian and thus all expectation values only depend on
\begin{equation}
    \hat{\gamma}(\vk) := \sum_{\alpha=1}^D \cos(k_\alpha)\,,
\end{equation}
i.e. for any operator $\hat{O}$ the following relation holds 
\begin{equation}
    \langle \hat{O}_{\vk} \rangle = \langle \hat{O}_{\vk'} \rangle =: \langle \hat{O}_{\gamma} \rangle\,,
\end{equation}
if $\hat{\gamma}(\vk) = \hat{\gamma}(\vk')$.
This allows us to replace the momentum sums with energy integrals using the bare density of states
\begin{equation}
    \rho(\gamma) := \frac{1}{N} \sum_{\vk} \delta \left(\gamma - \hat{\gamma} (\vk) \right)\,.
\end{equation}
As an example, consider 
\begin{equation}
    \Delta_\text{SC} = \frac{U}{2N} \int \mathrm{d}\gamma \rho(\gamma) \langle f_{\gamma} \rangle\,.
\end{equation}
We will be using the exact densities of states for the square and the simple cubic lattice found in Ref. \cite{Hanisch97}.
Solving the mean-field equations iteratively, one obtains the groundstate phase diagram at temperature $T=0$ as shown in \textcolor{red}{FIGURE}.
Note, that the phase-separated state as well as the $d_{x^2 - y^2}$-superconducting state, which has been confirmed in for the square lattice \cite{Micnas88b,Huang13},
cannot be described by our method. This is due to our restriction to terms that are proportional to $\hat{\gamma}(\vk)$.

%%%%%%%%%%%%%%%%%%%%%%%%%%%%%%%%%%%%%%%%%%%%%%%%%%%%%%%%%%%%%%%%%%%%%%%%%%%%%%%%%%%%%%%%%%%%%%%%%%%%%%%%%%%%%%%%%%%%%
%%%%%%%%%%%%%%%%%%%%%%%%%%%%%%%%%%%%%%%%%%%%%%%%%%%%%%%%%%%%%%%%%%%%%%%%%%%%%%%%%%%%%%%%%%%%%%%%%%%%%%%%%%%%%%%%%%%%%
%%%%%                                                  Results                                                  %%%%%
%%%%%%%%%%%%%%%%%%%%%%%%%%%%%%%%%%%%%%%%%%%%%%%%%%%%%%%%%%%%%%%%%%%%%%%%%%%%%%%%%%%%%%%%%%%%%%%%%%%%%%%%%%%%%%%%%%%%%
%%%%%%%%%%%%%%%%%%%%%%%%%%%%%%%%%%%%%%%%%%%%%%%%%%%%%%%%%%%%%%%%%%%%%%%%%%%%%%%%%%%%%%%%%%%%%%%%%%%%%%%%%%%%%%%%%%%%%

\section{Results}\label{sec:results}
 
...

%%%%%%%%%%%%%%%%%%%%%%%%%%%%%%%%%%%%%%%%%%%%%%%%%%%%%%%%%%%%%%%%%%%%%%%%%%%%%%%%%%%%%%%%%%%%%%%%%%%%%%%%%%%%%%%%%%%%%
%%%%%%%%%%%%%%%%%%%%%%%%%%%%%%%%%%%%%%%%%%%%%%%%%%%%%%%%%%%%%%%%%%%%%%%%%%%%%%%%%%%%%%%%%%%%%%%%%%%%%%%%%%%%%%%%%%%%%
%%%%%                                                Conclusion                                                 %%%%%
%%%%%%%%%%%%%%%%%%%%%%%%%%%%%%%%%%%%%%%%%%%%%%%%%%%%%%%%%%%%%%%%%%%%%%%%%%%%%%%%%%%%%%%%%%%%%%%%%%%%%%%%%%%%%%%%%%%%%
%%%%%%%%%%%%%%%%%%%%%%%%%%%%%%%%%%%%%%%%%%%%%%%%%%%%%%%%%%%%%%%%%%%%%%%%%%%%%%%%%%%%%%%%%%%%%%%%%%%%%%%%%%%%%%%%%%%%%

\section{Conclusion}\label{sec:conclusion}

...


%==============================================================================
% Acknowledgments 
%==============================================================================
\begin{acknowledgments} 
    ....
\end{acknowledgments}

\bibliography{sn-bibliography}
%\bibliography{../../A_bibinput/liter10.bib}
		
\end{document}
