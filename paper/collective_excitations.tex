%Version 2.1 April 2023
% See section 11 of the User Manual for version history
%
%%%%%%%%%%%%%%%%%%%%%%%%%%%%%%%%%%%%%%%%%%%%%%%%%%%%%%%%%%%%%%%%%%%%%%
%%                                                                 %%
%% Please do not use \input{...} to include other tex files.       %%
%% Submit your LaTeX manuscript as one .tex document.              %%
%%                                                                 %%
%% All additional figures and files should be attached             %%
%% separately and not embedded in the \TeX\ document itself.       %%
%%                                                                 %%
%%%%%%%%%%%%%%%%%%%%%%%%%%%%%%%%%%%%%%%%%%%%%%%%%%%%%%%%%%%%%%%%%%%%%

%%\documentclass[referee,sn-basic]{sn-jnl}% referee option is meant for double line spacing

%%=======================================================%%
%% to print line numbers in the margin use lineno option %%
%%=======================================================%%

%%\documentclass[lineno,sn-basic]{sn-jnl}% Basic Springer Nature Reference Style/Chemistry Reference Style

%%======================================================%%
%% to compile with pdflatex/xelatex use pdflatex option %%
%%======================================================%%

%%\documentclass[pdflatex,sn-basic]{sn-jnl}% Basic Springer Nature Reference Style/Chemistry Reference Style


%%Note: the following reference styles support Namedate and Numbered referencing. By default the style follows the most common style. To switch between the options you can add or remove �Numbered� in the optional parenthesis. 
%%The option is available for: sn-basic.bst, sn-vancouver.bst, sn-chicago.bst, sn-mathphys.bst. %  
 
%%\documentclass[sn-nature]{sn-jnl}% Style for submissions to Nature Portfolio journals
%%\documentclass[sn-basic]{sn-jnl}% Basic Springer Nature Reference Style/Chemistry Reference Style
\documentclass[iicol,sn-mathphys,Numbered]{sn-jnl}% Math and Physical Sciences Reference Style
%%\documentclass[sn-aps]{sn-jnl}% American Physical Society (APS) Reference Style
%%\documentclass[sn-vancouver,Numbered]{sn-jnl}% Vancouver Reference Style
%%\documentclass[sn-apa]{sn-jnl}% APA Reference Style 
%%\documentclass[sn-chicago]{sn-jnl}% Chicago-based Humanities Reference Style
%%\documentclass[default]{sn-jnl}% Default
%%\documentclass[default,iicol]{sn-jnl}% Default with double column layout

%%%% Standard Packages
%%<additional latex packages if required can be included here>

\usepackage{graphicx}%
\usepackage{multirow}%
\usepackage{amsmath,amssymb,amsfonts}%
\usepackage{amsthm}%
\usepackage{mathrsfs}%
\usepackage[title]{appendix}%
\usepackage{xcolor}%
\usepackage{textcomp}%
\usepackage{manyfoot}%
\usepackage{booktabs}%
\usepackage{algorithm}%
\usepackage{algorithmicx}%
\usepackage{algpseudocode}%
\usepackage{listings}%
%%%%

%%%%%=============================================================================%%%%
%%%%  Remarks: This template is provided to aid authors with the preparation
%%%%  of original research articles intended for submission to journals published 
%%%%  by Springer Nature. The guidance has been prepared in partnership with 
%%%%  production teams to conform to Springer Nature technical requirements. 
%%%%  Editorial and presentation requirements differ among journal portfolios and 
%%%%  research disciplines. You may find sections in this template are irrelevant 
%%%%  to your work and are empowered to omit any such section if allowed by the 
%%%%  journal you intend to submit to. The submission guidelines and policies 
%%%%  of the journal take precedence. A detailed User Manual is available in the 
%%%%  template package for technical guidance.
%%%%%=============================================================================%%%%

%\jyear{2021}%

%% as per the requirement new theorem styles can be included as shown below
\theoremstyle{thmstyleone}%
\newtheorem{theorem}{Theorem}%  meant for continuous numbers
%%\newtheorem{theorem}{Theorem}[section]% meant for sectionwise numbers
%% optional argument [theorem] produces theorem numbering sequence instead of independent numbers for Proposition
\newtheorem{proposition}[theorem]{Proposition}% 
%%\newtheorem{proposition}{Proposition}% to get separate numbers for theorem and proposition etc.

\theoremstyle{thmstyletwo}%
\newtheorem{example}{Example}%
\newtheorem{remark}{Remark}%

\theoremstyle{thmstylethree}%
\newtheorem{definition}{Definition}%

\raggedbottom
%%\unnumbered% uncomment this for unnumbered level heads

\begin{document}

\title[Article Title]{Collective excitations in the 2- and 3-dimensional extended Hubbard model}

%%=============================================================%%
%% Prefix	-> \pfx{Dr}
%% GivenName	-> \fnm{Joergen W.}
%% Particle	-> \spfx{van der} -> surname prefix
%% FamilyName	-> \sur{Ploeg}
%% Suffix	-> \sfx{IV}
%% NatureName	-> \tanm{Poet Laureate} -> Title after name
%% Degrees	-> \dgr{MSc, PhD}
%% \author*[1,2]{\pfx{Dr} \fnm{Joergen W.} \spfx{van der} \sur{Ploeg} \sfx{IV} \tanm{Poet Laureate} 
%%                 \dgr{MSc, PhD}}\email{iauthor@gmail.com}
%%=============================================================%%

\author*[1]{\fnm{Joshua} \sur{Alth\"user}}\email{joshua.althueser@tu-dortmund.de}

\author*[1]{\fnm{G\"otz S.} \sur{Uhrig}}\email{goetz.uhrig@tu-dortmund.de}

\affil*[1]{\orgdiv{Condensed matter theory}, \orgname{TU Dortmund}, \orgaddress{\street{Otto-Hahn-Straße 4}, \city{Dortmund}, \postcode{44227}, \state{North Rhine-Westphalia}, \country{Germany}}}



%%%%%%%%%%%%%%%%%%%%%%%%%%%%%%%%%%%%%%%%%%%%%%%%%%%%%%%%%%%%%%%%%%%%%%%%%%%%%%%%%%%%%%%%%%%%%%%%%%%%%%%%%%%%%%%%%%%%%
%%%%%%%%%%%%%%%%%%%%%%%%%%%%%%%%%%%%%%%%%%%%%%%%%%%%%%%%%%%%%%%%%%%%%%%%%%%%%%%%%%%%%%%%%%%%%%%%%%%%%%%%%%%%%%%%%%%%%
%%%%%                                                  Abstract                                                 %%%%%
%%%%%%%%%%%%%%%%%%%%%%%%%%%%%%%%%%%%%%%%%%%%%%%%%%%%%%%%%%%%%%%%%%%%%%%%%%%%%%%%%%%%%%%%%%%%%%%%%%%%%%%%%%%%%%%%%%%%%
%%%%%%%%%%%%%%%%%%%%%%%%%%%%%%%%%%%%%%%%%%%%%%%%%%%%%%%%%%%%%%%%%%%%%%%%%%%%%%%%%%%%%%%%%%%%%%%%%%%%%%%%%%%%%%%%%%%%%

\abstract{The abstract serves both as a general introduction to the topic and as a brief, non-technical summary of the main results and their implications. 
Authors are advised to check the author instructions for the journal they are submitting to for word limits and if structural elements like subheadings, citations, or equations are permitted.}


\keywords{Hubbard model, collective excitations, superconductivity, charge density wave, antiferromagnetism}

%%\pacs[JEL Classification]{D8, H51}

%%\pacs[MSC Classification]{35A01, 65L10, 65L12, 65L20, 65L70}

\maketitle

%%%%%%%%%%%%%%%%%%%%%%%%%%%%%%%%%%%%%%%%%%%%%%%%%%%%%%%%%%%%%%%%%%%%%%%%%%%%%%%%%%%%%%%%%%%%%%%%%%%%%%%%%%%%%%%%%%%%%
%%%%%%%%%%%%%%%%%%%%%%%%%%%%%%%%%%%%%%%%%%%%%%%%%%%%%%%%%%%%%%%%%%%%%%%%%%%%%%%%%%%%%%%%%%%%%%%%%%%%%%%%%%%%%%%%%%%%%
%%%%%                                                Introduction                                               %%%%%
%%%%%%%%%%%%%%%%%%%%%%%%%%%%%%%%%%%%%%%%%%%%%%%%%%%%%%%%%%%%%%%%%%%%%%%%%%%%%%%%%%%%%%%%%%%%%%%%%%%%%%%%%%%%%%%%%%%%%
%%%%%%%%%%%%%%%%%%%%%%%%%%%%%%%%%%%%%%%%%%%%%%%%%%%%%%%%%%%%%%%%%%%%%%%%%%%%%%%%%%%%%%%%%%%%%%%%%%%%%%%%%%%%%%%%%%%%%

\section{Introduction}\label{sec:introduction}

The Hubbard model has been employed in a plethora of previous studies. 
Early studies proved the existence of eigenstates to the Hubbard Hamiltonian that exhibit off-diagonal long-range order,
tantalizing the model's usage for the description of high-temperature superconductivity \cite{yang89}.
Shortly afterward, an exact $SO(4)$ symmetry was discovered,
which allows a degeneracy of superconductivity (SC) and a charge-density wave (CDW) governed by an attractive on-site interaction \cite{yang90}.
This coexistence can be argued by the existence of a particle-hole transformation on bipartite lattices, 
that maps the attractive Hubbard model exactly onto the repulsive one, exhibiting antiferromagnetism (AFM) \cite{Hirsch85}.
The previously mentioned SC and CDW phases map to different spin operators \cite{zitko15,lieb89}.
\newline
Many recent experimental and theoretical studies examined the driven systems in the hope of inducing superconductivity 
\cite{Nicoletti14,Krull14,Moor14,Casandruc15,patel16,sentef17,Buenemann17}.
Other studies, focusing on equilibrium systems, 
investigated the phases and various quantities therein of said model including various additional interactions with and without doping 
\cite{Micnas88,Micnas88b,Micnas89,Dzierzawa92,Kostyrko92,Eriksson95,Staudt00,Onari04,Toschi05,Brackett16,Paki19,romer20,Sushchyev22}.
In this paper, we will restrict ourselves to the half-filled Hubbard model, including an additional intersite interaction, on a square and a simple cubic lattice.
\textcolor{red}{Für den 3D Fall finde ich keine Phasendiagramme des extended Hubbard models, ggf nochmal suchen...}
The 2D case already exhibits a wide range of possible phases, including CDW, AFM as well as $s$- and $d_{x^2 - y^2}$-wave superconductivity
\cite{Micnas88b,Tsuchiura95,Su01,Su04,ha11,Huang13,Jiang22}.
\newline
In this article, we will first employ a mean-field approximation to the interaction terms 
and then use the methodology from the so-called iterated equations of motion approach,
which has already seen success in the handling of interaction quenches \cite{uhrig09,hamerla13,hamerla14,bleicker18}.
Here, one commutes operators from a suitable operator basis with the Hamiltonian and adds the newly occurring terms to it.
Naturally, a truncation is necessary for most practical applications, however, the approximation becomes better the more terms are included within the basis.
The functionality of this method has been compared to the density matrix formalism and has proved to be considerably more accurate \cite{Kalthoff17}.
Additionally, we will show an explicit way to obtain various Green's functions from the aforementioned methodology.
By extension, we also obtain the spectral functions of the investigated systems and discuss the excitations found therein.
The prominent examples in the SC phase are the well-known phase (Leggett) and amplitude (Higgs) modes \cite{Cea14,Krull16,Schwarz20,Fan22}.
\newline
The remainder of this paper is organized as follows:
In \autoref{sec:model} we will introduce the model and its Hamiltonian as well as the main equations and relations used throughout this article.
Next, \autoref{sec:results} is split into two parts, \autoref{ssec:square} and \autoref{ssec:simple_cubic}, 
in which we will show the results for the square and the simple cubic lattice respectively.
\textcolor{red}{Alternative Aufteilung: Übergänge SC-CDW-Phase und AFM-CDW-Phase, da sich das für beide Gitter ja doch sehr ähnlich verhält, wäre das wohl weniger Doppelung.}
Lastly, in \autoref{sec:conclusion}, we will discuss our results.
In the appendix \textcolor{red}{(or in the supplemental material?)}, we will derive and explain our methods in detail.

\section{Model and governing equations}\label{sec:model}

The Hamiltonian of the extended Hubbard model is given by
\begin{equation}
    \begin{aligned}
        H = &-t \sum_{\langle i, j \rangle, \sigma} \left( c_{i\sigma}^\dagger c_{j\sigma} + \text{h.c.} \right) 
        + \mu \sum_{i,\sigma} n_{i\sigma} \\
        &+ U \sum_{i} n_{i\uparrow} n_{i\downarrow} 
        + \frac{V}{2} \sum_{\langle i, j\rangle, \sigma} n_{i\sigma} n_{j\sigma}\,,
    \end{aligned}
\end{equation}
where $c_{i\sigma}^{(\dagger)}$ annihilates (creates) an electron with spin $\sigma$ on lattice site $i$ and $\langle i, j\rangle$ denotes the summation over next neighbours.
The parameters are the hopping amplitude $t$, the onsite interaction $U$,the intersite interaction $V$ and the chemical potential $\mu$.
Throughout this article, we will give all energy related quantities in terms of the hopping amplitude, which we subsequently set to unity $t=1$.
Applying a Fourier transformation into $k$-space yields the single-particle dispersion 
\begin{equation}
    \epsilon_0 (\vec{k}) = -2t \sum_{\alpha=1}^D \cos(k_\alpha)\,,\quad k_\alpha \in [-\pi, \pi)\,,
\end{equation}
with the system's dimension $D$ and the dimensionless momentum $\vec{k}$.
The interaction terms are then mean-field decoupled. 
We follow the naming scheme of Ref. \cite{sentef17}
\begin{equation}
    \begin{aligned}
        n_{k\sigma} &:= c_{\vec{k}\sigma}^\dagger c_{\vec{k}\sigma}\,,          &f_k     &:= c_{-\vec{k}\downarrow} c_{\vec{k}\uparrow}\,, \\
        g_{k\sigma} &:= c_{\vec{k}\sigma}^\dagger c_{\vec{k}+\vec{Q}\sigma}\,,  &\eta_k  &:= c_{-\vec{k}-\vec{Q}\downarrow} c_{\vec{k}\uparrow}\,,
    \end{aligned}
\end{equation}
where $\vec{Q} := (\pi, ...)$ defines the nesting vector for the CDW and AFM phases.
We use these abbreviations to write down the mean-field parameters
\begin{subequations}
    \begin{align}
        \Delta_\text{CDW} &= \left(\frac{U}{2N} - \frac{ZV}{N}\right) \sum_{\vec{k}\sigma} \langle g_{k\sigma} \rangle\,, \\
        \Delta_\text{AFM} &= \frac{U}{2N} \sum_{\vec{k}} \left( \langle g_{k\uparrow} \rangle - \langle g_{k\downarrow} \rangle \right)\,, \\
        \Delta_\text{SC} &= \frac{U}{N} \sum_{\vec{k}} \langle f_k \rangle\,, \\
        \Delta_\eta &= \frac{U}{N} \sum_{\vec{k}} \langle \eta_k \rangle\,, \\
        \Delta_n &= \frac{V}{N} \sum_{\vec{k},\sigma} \sum_{\alpha=1}^D \cos k_\alpha \langle n_{k\sigma} \rangle\,,
    \end{align}
\end{subequations}
where $Z$ denotes the coordination number of the lattice.
In total, we obtain the mean-field Hamiltonian in spinor representation as
\begin{equation}
    H_\text{MF} = \sum_{\vec{k}} \Psi^\dagger (\vec{k}) h(\vec{k}) \Psi (\vec{k})\,,
\end{equation}
with the spinors \textcolor{red}{Wie macht man denn diese Ganzzeilengleichungen?}
%\begin{strip}
\begin{equation}
    h(\vec{k}) .= \begin{pmatrix}
        \epsilon_0 (\vec{k}) & \Delta_\text{CDW}^* - \Delta_\text{AFM}^* & \Delta_\text{SC} & \Delta_\eta \\
        \Delta_\text{CDW} - \Delta_\text{AFM} & \epsilon_0 (\vec{k} + \vec{Q}) & \Delta_\eta & \Delta_\text{SC} \\
        \Delta_\text{SC}^* & \Delta_\eta^* & - \epsilon_0 (-\vec{k}) & - \Delta_\text{CDW} - \Delta_\text{AFM} \\
        \Delta_\eta^* & \Delta_\text{SC}^* & - \Delta_\text{CDW}^* - \Delta_\text{AFM}^* & - \epsilon_0 (-\vec{k} - \vec{Q})
        \end{pmatrix}
\end{equation}
%\end{strip}
Now, the entire Hamiltonian and thus all expectation values only depend on
\begin{equation}
    \hat{\gamma}(\vec{k}) := \sum_{\alpha=1}^D \cos(k_\alpha)\,,
\end{equation}
which allows us to replace the momentum sums by energy integrals using the density of states
\begin{equation}
    \rho(\gamma) := \frac{1}{N} \sum_{\vec{k}} \delta \left(\gamma - \hat{\gamma} (\vec{k}) \right)\,.
\end{equation}
As an example, consider 
\begin{equation}
    \Delta_\text{SC} = \frac{U}{2N} \int \mathrm{d}\gamma \rho(\gamma) \langle f_{\gamma} \rangle\,.
\end{equation}
We will be using the exact densities of states for the square and the simple cubic lattice found in Ref. \cite{Hanisch97}.
Solving the mean field equations iteratively, one obtains the groundstate phase diagram at temperature $T=0$ as shown in \textcolor{red}{FIGURE}.
Note, that the phase separated state as well as the $d_{x^2 - y^2}$-superconducting state, that have been confirmed in for the square lattice \cite{Micnas88b,Huang13},
cannot be described by our method. This is due to our restriction to terms that are proportional to $\hat{\gamma}(\vec{k})$.



\section{Results}\label{sec:results}

\subsection{Square lattice}\label{ssec:square}

Ergebnisse des Quadratgitters

\subsection{Simple cubic lattice}\label{ssec:simple_cubic}

Ergebnisse des kubischen Gitters

\section{Conclusion}\label{sec:conclusion}

Conclusions may be used to restate your hypothesis or research question, restate your major findings, explain the relevance and the added value of your work, highlight any limitations of your study, describe future directions for research and recommendations. 

In some disciplines use of Discussion or 'Conclusion' is interchangeable. It is not mandatory to use both. Please refer to Journal-level guidance for any specific requirements. 

\backmatter

\bmhead{Supplementary information}

If your article has accompanying supplementary file/s please state so here. 

Authors reporting data from electrophoretic gels and blots should supply the full unprocessed scans for key as part of their Supplementary information. This may be requested by the editorial team/s if it is missing.

Please refer to Journal-level guidance for any specific requirements.

\bmhead{Acknowledgments}

Acknowledgments are not compulsory. Where included they should be brief. Grant or contribution numbers may be acknowledged.

Please refer to Journal-level guidance for any specific requirements.

\section*{Declarations}

Some journals require declarations to be submitted in a standardised format. Please check the Instructions for Authors of the journal to which you are submitting to see if you need to complete this section. If yes, your manuscript must contain the following sections under the heading `Declarations':

\begin{itemize}
\item Funding
\item Conflict of interest/Competing interests (check journal-specific guidelines for which heading to use)
\item Ethics approval 
\item Consent to participate
\item Consent for publication
\item Availability of data and materials
\item Code availability 
\item Authors' contributions
\end{itemize}

\noindent
If any of the sections are not relevant to your manuscript, please include the heading and write `Not applicable' for that section. 

%%===================================================%%
%% For presentation purpose, we have included        %%
%% \bigskip command. please ignore this.             %%
%%===================================================%%
\bigskip
\begin{flushleft}%
Editorial Policies for:

\bigskip\noindent
Springer journals and proceedings: \url{https://www.springer.com/gp/editorial-policies}

\bigskip\noindent
Nature Portfolio journals: \url{https://www.nature.com/nature-research/editorial-policies}

\bigskip\noindent
\textit{Scientific Reports}: \url{https://www.nature.com/srep/journal-policies/editorial-policies}

\bigskip\noindent
BMC journals: \url{https://www.biomedcentral.com/getpublished/editorial-policies}
\end{flushleft}

\begin{appendices}

\section{Section title of first appendix}\label{secA1}

An appendix contains supplementary information that is not an essential part of the text itself but 
which may be helpful in providing a more comprehensive understanding of the research problem or it is information that is too cumbersome to be included in the body of the paper.


\end{appendices}

\bibliography{sn-bibliography}

\end{document}
